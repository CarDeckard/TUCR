%built list of things I did
\documentclass{article}

\title{Detailed Instructions Into Further Bitmap Updating and Manipulation}
\date{2019 \\ August }
\author{Carson Deckard \and Collin Lehrian \and Josiah McClurg \and Shane Wozniak \\ Computer Science & Engineering Department \\ Taylor University}

\begin{document}
\maketitle

%
%The motivation for why our research is worth looking at 
%
\section{Introduction}
\justify

Data warehouses have become an integral aspect of large-scale computing. Corporations such as Google, Amazon, and Apple use these massive warehouses of servers to store and compute queries on their data. Of course, this computing capability comes at a cost. Data warehouses need to stay cool to utilize the full computing capabilities due to heat limitations of CPUs. We have seen that a majority of data warehouses' running costs are cooling [Citation hopefully].
%Would like some feedback on this paragraph and possible citations that are %needed still. -Collin
There are various ways to deal with the heat from the computers. One would be more efficient cooling which could be a large upfront cost. Software changes could improve efficiency and storage while fixing heat problems without the need for improved cooling units.     
%
%the background knowledge that will help others to understand some of the paper 
%
\section{Background}
\justify


In order to get the most out of this paper two things have to be understood; the basics of bitmaps and the basics of WAH compression. \par

Bitmaps are two dimensional arrays where the rows represent tuples and the columns represent bins of binary vectors where a “1” represents that the tuple value is at that index location. This enables the ability to perform very fast logical operations onto a single column or group of columns rather than the entire dataset. For example, say we have a data type that has a number as well as a color. \par

For example, we will use the list 3 Green, 4 Blue, 2 Red, 4 Red, 2 Blue. This would create a bitmap as featured in Fig. 1. From this we can find all Red numbers greater than 2 by using an AND of bins 3 and 4 as well as an XAND with the Red bin. \par

%image bitmap.png 

Word Aligned Hybrid (WAH) is a popular compression strategy used often as the benchmark against newly created compression strategies. WAH compression is comprised of two types of words; literal and fill. WAH works by reading in a whole word. For 32-bit words the literal words would store 31-bits and the fill words would be in multiples of 31. For 64-bit words literal words would store 63-bit and fill words would be multiples of 63. This is done because the most significant bit (MSB) represents whether the compressed word is a literal word or a fill word. For 32-bit words, the MSB of literal words is set to 0 followed by the rest of the 31 bits. Fill words have their MSB set to 1. The following bit denotes whether the fill word is of 1’s or 0’s. The remaining 30-bits represent how many multiples of 31 there are. \par

To show WAH compression we will first show a 128-bit hexadecimal number being compressed into both literal words and a fill word (Fig. 2). We will then feature two 32-bit WAH compressed hexadecimal numbers being AND’ed together (Fig. 3). As you’ll notice A starts out with a run of 1’s. Anything AND’ed against a 1 is just itself so we can just copy over the two first words of B. B than has a fill word that represents two runs of 0’s. As we can recall anything AND’ed with 0 is going to be 0. This means that we can just copy down the fill word into our answer. Both of our last words are just literal words that can be AND’ed together without any special case. \par

%image wahcompression.png

%image A&B.png

RoaringBitmap is another popular compression strategy and the second compression algorithm that we compared. Roaring and WAH at their core set out to achieve the same goal. They are meant to compress bitmaps to save space in memory for large datasets. WAH uses RLE (Run-Length Encoding) to achieve this compression while Roaring is a hybrid compression that merges a sorted list and bitmap encoding. \par

%
%The steps that we took to get where we are
%
\section{Methods}
\justify

%
%What we got out of our research 
%
\section{Results}
\justify

%
%what our results mean for the community
%
\section{Discussion}
\justify

%
%what can be done moving forward
%
\section{Conclusion & Future Works}
\justify

%
%who gave us moneys
%
\section{Acknowledgements}
\justify

We would like to thank FMUS as well as The Women’s Giving Circle for allowing us to financially continue our research as well as grant us with the opportunity to pursue our  academic growth. We would also like to thank Taylor University for allowing us access to their labs and resources. \par

%
%who gave us  knowledges
%
\section{References}
\justify

\end{document}